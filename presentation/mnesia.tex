\documentclass[10pt]{beamer}
\usepackage{fontspec}
\usepackage{listings}

\setmainfont{Ubuntu}[]
\setsansfont{Ubuntu}[]
\setmonofont{Ubuntu Mono}[]

% https://hartwork.org/beamer-theme-matrix/
\usetheme{Singapore}
\usecolortheme{dove}
\beamertemplatenavigationsymbolsempty
\setbeamertemplate{headline}{}

\lstset{
  language=ML,
  keywordstyle=\color{blue},
  backgroundcolor=\color{lightgray}
}

\title{Использование базы данных Mnesia в чат-сервере}
\author{Юра Жлоба}
\institute{Wargaming.net}
\date{Май 2019}

\begin{document}
\maketitle

\section{Интро}

\begin{frame}
\frametitle{Mnesia}
\centering
Распределенная key value база данных,
\par \bigskip
встраиваемая в Erlang приложения.
\end{frame}

\begin{frame}
\frametitle{Mnesia}
\centering
Почему ее не рекомендуют использовать?
\par \bigskip
И почему все же используют?
\par \bigskip
Как она используется в чат-сервере?
\end{frame}

\begin{frame}
\frametitle{Amnesia}
\centering
1999 год
\par \bigskip
Изначально база данных называлась Amnesia.
\par \bigskip
Это название не понравилось кому-то из менеджмента.
\par \bigskip
\textcolor{gray}{"So we dropped the A, and the name stuck."} Joe Armstrong.
\end{frame}

\begin{frame}
\frametitle{Amnesia}
\centering
Традицию продолжила компания WhatsApp.
\par \bigskip
Они назвали свою БД
\par \bigskip
\textbf{ForgETS}.
\end{frame}

\section{Фичи}

\begin{frame}
\frametitle{Фичи}
\begin{itemize}[<+->]
\item Работает внутри эрланговской ноды,\newline
  \textcolor{gray}{не нужно передавать данные по сети.}
\item Хранит данные нативно (Erlang term),\newline
  \textcolor{gray}{не нужно сериализовать/десериализовать данные.}
\item Хранит данные в ETS/DEST таблицах,\newline
  \textcolor{gray}{чтение и запись работают очень быстро.}
\end{itemize}
\end{frame}

\begin{frame}
\frametitle{Фичи}
\centering
Работает внутри эрланговского кластера.
\par
\bigskip
Данные доступны отовсюду в кластере \textcolor{gray}{(сетевая прозрачность)}.
\par
\bigskip
Полная реплика данных на каждой ноде.
\end{frame}

\begin{frame}
\frametitle{Фичи}
\begin{itemize}
\item Транзакции \textcolor{gray}{(ACID)}.
\item Вторичные индексы.
\item Миграции \textcolor{gray}{(структуры таблиц и данных)}.
\item Шардинг \textcolor{gray}{(fragmented tables)}.
\end{itemize}
\end{frame}

\section{API}

\begin{frame}
\frametitle{API}
\begin{itemize}[<+->]
\item Базовые KV операции:\newline
  \textcolor{gray}{read, write, delete.}
\item ETS/DETS API:\newline
  \textcolor{gray}{lookup, match, select.}
\item Fold:\newline
  \textcolor{gray}{foldl, foldr.}
\item QLC\newline
  \textcolor{gray}{Query List Comprehension.}
\end{itemize}
\end{frame}

\begin{frame}[fragile]
\frametitle{Query List Comprehension}
\begin{lstlisting}
qlc:q([X || X <- mnesia:table(shop)])
\end{lstlisting}
\begin{lstlisting}
qlc:q([
    Xshop.item || X <- mnesia:table(shop),
    Xshop.quantity < 250
])
\end{lstlisting}
\begin{lstlisting}
qlc:q([
    Xshop.item ||
    X <- mnesia:table(shop),
    Xshop.quantity < 250,
    Y <- mnesia:table(cost),
    Xshop.item =:= Ycost.name,
    Ycost.price < 2
])
\end{lstlisting}
\end{frame}

\begin{frame}
\frametitle{Транзакции}
\centering
Синхронные и "обыкновенные".
\par \bigskip
Pessimistic locking.
\par \bigskip
Медленные.
\par \bigskip
Но без них нет консистентности данных.
\end{frame}

\section{Mnesia как распределенная система}

\begin{frame}
\frametitle{Консистентность данных}
\begin{itemize}[<+->]
\item Транзакции работают через 2PC.
\item Strict quorum protocol,\newline
  \textcolor{gray}{все ноды должны подтвердить транзакцию.}
\item Гибкие настройки репликации,\newline
  \textcolor{gray}{можно явно указать, на каких нодах и как хранить данные.}
\item Неплохо переживает рестарты нод в кластере.
\item Плохо переживает network partition.
\end{itemize}
\end{frame}

\begin{frame}
\frametitle{С точки зрения CAP теоремы}
\begin{itemize}[<+->]
\item Это CA система,\newline
  \textcolor{gray}{не устойчива к network partition (P).}
\item Можно повысить A отказавшись от C.\newline
  \textcolor{gray}{dirty mode, без транзакций.}
\item А если мне вообще не нужна репликация на несколько нод?\newline
  \textcolor{gray}{Тогда просто бери ETS/DETS.}
\end{itemize}
\end{frame}

\section{Репутация Mnesia}

\begin{frame}
\frametitle{Репутация Mnesia}
\centering
Мнение широко известных в узких кругах авторитетов.
\par
\bigskip
Печальный опыт с персистентными очередями в RabbitMQ.
\par
\bigskip
Слухи из Стокгольма от местных разработчиков.
\end{frame}

\begin{frame}
\frametitle{Репутация Mnesia}
\centering
Суть проблемы в том,
\par \bigskip
что если нода не была корректно остановлена, а упала,
\par \bigskip
то восстановление большой таблицы с диска может занять часы.
\end{frame}

\begin{frame}
\frametitle{Репутация Mnesia}
\centering
\textcolor{red}{Downtime сервиса может длится несколько часов!}
\par \bigskip
На этом про Mnesia можно было бы забыть и не вспоминать,
\par \bigskip
но ...
\end{frame}

\section{Применение Mnesia}

\begin{frame}
\frametitle{Применение Mnesia}
\centering
Но её можно применить с пользой.
\end{frame}

\begin{frame}
\frametitle{Задача}
\centering
Кластер из нескольких эрланг-нод.
\par \bigskip
Нужно хранить пользовательские сессии,
\par \bigskip
так, чтобы они были доступны во всех нодах кластера.
\end{frame}

\begin{frame}
\frametitle{Прежнее решение}
\begin{itemize}[<+->]
\item Сессии хранятся в MySQL.
\item Данные консистентны и доступны все нодам.
\item Latency больше, чем могло бы быть.
\end{itemize}
\end{frame}

\begin{frame}
\frametitle{Задача}
\begin{itemize}[<+->]
\item Конечно, хочется иметь эту инфу прямо в ноде.
\item Кешировать в ETS?
\item Хорошо, а как обновить этот кэш на всех нодах?
\item Вот если бы был распределенный кэш ...
\item Постойте-ка, а Mnesia -- это что?
\end{itemize}
\end{frame}

\begin{frame}
\frametitle{Mnesia не вызывает проблем, если:}
\begin{itemize}
\item Не нужно персистентное хранение данных.
\item Не нужны сложные запросы с транзакциями.
\item Данные относительно дешево реплицируются.
\end{itemize}
\end{frame}

\begin{frame}
\frametitle{С Mnesia будут проблемы, если:}
\begin{itemize}
\item Нужно хранить много данных.
\item Нужно хранить их персистентно.
\item Объем данных постоянно растет.
\item Выполняются сложные запросы к данным.
\end{itemize}
\end{frame}

\begin{frame}
\frametitle{Применение Mnesia}
\centering
Все это -- типичные сценарии использования типичной БД.
\par \bigskip
И все это -- плохо для Mnesia.
\end{frame}

\begin{frame}
\frametitle{Применение Mnesia}
\centering
Идеальный сценарий для Mnesia:
\par \bigskip
in-memory хранение пользовательских сессий.
\end{frame}

\begin{frame}
\frametitle{Применение Mnesia}
\centering
В такой роли ее используют:
\par \bigskip
WhatsUp \textcolor{gray}{(на ранних этапах)},
\par \bigskip
League of Legends Chat,
\par \bigskip
Discord,
\par \bigskip
Ejabberd.
\end{frame}

\begin{frame}
\frametitle{Применение Mnesia}
\centering
\textbf{Mnesia -- это не БД, это кэш :)}
\end{frame}

\begin{frame}
\frametitle{Еще раз про ключевые преимущества}
\begin{itemize}
\item Данные прямо в памяти ноды,\newline
  \textcolor{gray}{за ними не надо ходить по сети}.
\item Данные в нативном виде,\newline
  \textcolor{gray}{их не надо сериализовать/десериализовать}.
\item Прозрачная репликация на все ноды кластера.
\end{itemize}
\end{frame}

\begin{frame}
\frametitle{Что важно для нас}
\begin{itemize}
\item Mnesia неплохо переживает рестарты\newline отдельных нод в кластере.
\item Потому что мы именно так обновляем кластер.
\item Но нужно знать объем данных и время их репликации.
\item Это этого зависит время downtime ноды при рестарте.
\end{itemize}
\end{frame}

\begin{frame}
\centering
Вопросы?
\end{frame}

\end{document}
