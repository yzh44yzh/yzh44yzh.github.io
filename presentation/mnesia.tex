\documentclass[9pt]{beamer}
\usepackage{fontspec}
\usepackage{listings}

\setmainfont{Ubuntu}[]
\setsansfont{Ubuntu}[]
\setmonofont{Ubuntu Mono}[]

% https://hartwork.org/beamer-theme-matrix/
\usetheme{Singapore}
\usecolortheme{dove}
\beamertemplatenavigationsymbolsempty
\setbeamertemplate{headline}{}

\title{Использование базы данных Mnesia в чат-сервере}
\author{Юра Жлоба}
\institute{Wargaming.net}
\date{Май 2019}

\begin{document}
\maketitle

\section{Интро}

\begin{frame}
  \frametitle{Mnesia}
  \framesubtitle{Распределенная key value база данных, встраиваемая в Erlang приложения.}
  Разберемся, почему ее не рекомендуют использовать и почему все же используют.\newline
  И узнаем, как она используется в чат-сервере.
\end{frame}

\begin{frame}
  \frametitle{Amnesia}
1999 год
Изначально база данных называлась Amnesia
Это название не понравилось кому-то из менеджмента
"So we dropped the A, and the name stuck." Joe Armstrong.
\end{frame}

\begin{frame}
  \frametitle{Amnesia}
  \begin{itemize}
  \item Традицию продолжила компания WhatsApp
    \pause
  \item Они назвали свою БД
    \pause
  \item ForgETS
  \end{itemize}
\end{frame}

\section{Фичи}

\begin{frame}
  \frametitle{Фичи}
  \begin{itemize}[<+->]
  \item Работает внутри эрланговской ноды\newline(Не нужно передавать данные по сети)
  \item Хранит данные нативно (Erlang term)\newline(Не нужно сериализовать/десериализовать данные)
  \item Хранит данные в ETS/DEST таблицах\newline(Чтение и запись работают очень быстро)
  \end{itemize}
\end{frame}

\begin{frame}
 Работает внутри эрланговского кластера
Данные доступны отовсюду в кластере (сетевая прозрачность)
Полная реплика данных на каждой ноде
\end{frame}

\begin{frame}
 Фичи
Транзакции (ACID)
Вторичные индексы
Миграции (структуры таблиц и данных)
Шардинг (fragmented tables)
\end{frame}

\begin{frame}
 С точки зрения CAP теоремы
Тут есть нюансы
 Если с транзакциями, то CP
И это медленно (очень)
 Если в dirty режиме, то AP
И тут никаких гарантий Consistency, даже "eventualy"
 А если я хочу CA?
Тогда просто бери ETS/DETS
\end{frame}

\section{API}

\begin{frame}
 API
 Базовые KV операции
read, write, delete
 ETS/DETS API
lookup, match, select
 Fold
foldl, foldr
 QLC
Query List Comprehension
\end{frame}

\begin{frame}[fragile]
\begin{lstlisting}[language=Erlang]
qlc:q([X || X <- mnesia:table(shop)])

qlc:q([
    Xshop.item || X <- mnesia:table(shop),
    Xshop.quantity < 250
])

qlc:q([
    Xshop.item ||
    X <- mnesia:table(shop),
    Xshop.quantity < 250,
    Y <- mnesia:table(cost),
    Xshop.item =:= Ycost.name,
    Ycost.price < 2
])
\end{lstlisting}
\end{frame}

\begin{frame}
 Транзакции
Синхронные и "обыкновенные"
Pessimistic locking
Медленные
Но без них нет консистентности данных
\end{frame}

\section{Репутация Mnesia}

\begin{frame}
 Репутация Mnesia
Мнение широко известных в узких кругах авторитетов
Печальный опыт с персистентными очередями в RabbitMQ
Слухи из Стокгольма от местных разработчиков
\end{frame}

\begin{frame}
 Репутация Mnesia
Суть проблемы в том,
что если нода не была корректно остановлена, а упала,
то восстановление большой таблицы с диска может занять часы.
\end{frame}

\begin{frame}
 Репутация Mnesia
Downtime сервиса может длится несколько часов!
На этом про Mnesia можно было бы забыть и не вспоминать
но ...
\end{frame}

\section{Применение Mnesia}

\begin{frame}
 Применение Mnesia
её можно применить с пользой
\end{frame}

\begin{frame}
 Задача
Кластер из нескольких эрланг-нод.
Нужно хранить пользовательские сессии,
так, чтобы они были доступны во всех нодах кластера.
\end{frame}

\begin{frame}
 Задача
Прежнее решение:
Сессии хранятся в MySQL
\end{frame}

\begin{frame}
 Задача
Конечно, хочется иметь эту инфу прямо в ноде.
Кешировать в ETS?
Хорошо, а как обновить этот кэш на всех нодах?
Вот если бы был распределенный кэш ...
Постойте-ка, а Mnesia -- это что?
\end{frame}

\begin{frame}
 Применение Mnesia
Mnesia не вызывает проблем, если:
не нужно персистентное хранение данных
не нужны сложные запросы с транзакциями
данные относительно дешево реплицируются (то есть, их не много)
\end{frame}

\begin{frame}
 Применение Mnesia
in-memory хранение пользовательских сессий
идеальный сценарий для Mnesia
\end{frame}

\begin{frame}
 Применение Mnesia
Mnesia не стоит использовать, если:
Нужно хранить много данных
Нужно хранить их персистентно
Объем данных постоянно растет
Выполняются сложные запросы к данным
\end{frame}

\begin{frame}
 Применение Mnesia
Все это -- типичные сценарии использования типичной БД
И все это -- плохо для Mnesia
\end{frame}

\begin{frame}
 Применение Mnesia
Mnesia -- это не БД, это кэш :)
\end{frame}

\begin{frame}
 Применение Mnesia
Еще раз про ключевые преимущества:
Данные прямо в памяти ноды, за ними не надо ходить по сети
Данные в нативном виде, их не надо сериализовать/десериализовать
Прозрачная репликация на все ноды кластера
\end{frame}

\begin{frame}
 Применение Mnesia
Что важно для нас:
Mnesia неплохо переживает рестарты отдельных нод в кластере
Потому что мы именно так обновляем кластер
Но нужно знать объем данных и время их репликации
Это этого зависит время downtime ноды при рестарте
\end{frame}

\begin{frame}
 Вопросы
\end{frame}

\end{document}
