\documentclass[10pt]{beamer}
\usepackage{fontspec}
\usepackage{listings}

\setmainfont{Ubuntu}[]
\setsansfont{Ubuntu}[]
\setmonofont{Ubuntu Mono}[]

\usetheme{Singapore}
\usecolortheme{dove}
\beamertemplatenavigationsymbolsempty
\setbeamertemplate{headline}{}

\lstset{
  language=ML,
  keywordstyle=\color{blue},
  backgroundcolor=\color{lightgray}
}

\title{Эрланг и Эликсир \\
  \large Зачем они вообще нужны? \\
  И зачем они нужны лично вам?}
\author{Юра Жлоба}
\institute{Wargaming.net}
\date{Ноябрь 2019}

\begin{document}
\maketitle

\begin{frame}
\frametitle{Агенда}
\centering
Чуть-чуть истории.
\par \bigskip
Чем хороша виртуальная машина.
\par \bigskip
Какие основные фишки у Эрланг.
\par \bigskip
Какие основные фишки у Эликсир.
\par \bigskip
Разбор мнений об Эликсир, бытующих в Руби-сообществе.
\end{frame}


\section{Чуть-чуть истории.}

{
\setbeamercolor{background canvas}{bg=orange}
\begin{frame}
\frametitle{Чуть-чуть истории}
\centering
Это важно для понимания сути.
\end{frame}
}

\begin{frame}
\frametitle{Агнер Краруп Эрланг}
\centering
Датский математик, статистик и инженер,
\par \bigskip
автор "Теории массового обслуживания".
\end{frame}

\begin{frame}
\frametitle{Теория массового обслуживания}
\centering
1909
\par \bigskip
Теория очередей, Queueing theory
\par \bigskip
Математическая модель для оценки пропускной способности телекоммуникационных сетей
\end{frame}

\begin{frame}
\frametitle{Теория массового обслуживания}
\begin{itemize}
\item не только сетей, но и
\item дорог (автомобильных, железнодорожных и т.д.)
\item больниц
\item складов, магазинов
\end{itemize}
\end{frame}

\begin{frame}
\frametitle{Эрланг, это}
\begin{itemize}
\item датский ученый
\item единица пропускной способности сети
\item язык программирования
\end{itemize}
\end{frame}

\begin{frame}
\frametitle{highload 80-х}
\centering
Эрикссон (Ericsson)
\par \bigskip
телекомуникационное оборудование и услуги.
\end{frame}

\begin{frame}
\frametitle{highload 80-х}
\begin{itemize}
\item сложное оборудование
\item сложный софт
\item большой траффик
\item жесткие требования по доступности сервиса
\end{itemize}
\end{frame}

\begin{frame}
\frametitle{Совсем краткая история}
\begin{itemize}
\item 80-е -- разработка языка
\item 90-е -- использование внутри компании Эрикссон
\item 2000-е -- выход в мир
\end{itemize}
\end{frame}

\begin{frame}
\frametitle{2007 год, закон Мура больше не работает}
\centering
Предел наращивания частот процессоров.
\par \bigskip
Рост количества процессоров и их ядер.
\par \bigskip
Необходимость в разработке многопоточных программ.
\end{frame}

\begin{frame}
\frametitle{2007 год, закон Мура больше не работает}
\centering
Рост интереса к ФП.
\par \bigskip
Копирование идей ФП в мейнстримовые языки.
\end{frame}

\begin{frame}
\frametitle{2011 год, появление Эликсир}
\centering
Жозе Валим (José Valim)
\par \bigskip
Один из основных разработчиков Ruby on Rails.
\end{frame}


\section{Чем хороша виртуальная машина.}

{
\setbeamercolor{background canvas}{bg=orange}
\begin{frame}
\frametitle{Чем хороша виртуальная машина.}
\centering
Всем!
\end{frame}
}

\begin{frame}
\frametitle{Erlang VM}
\centering
Представляет собой операционную систему в миниатюре:
\par \bigskip
планировщик процессов, управление памятью,
\par
дисковый и сетевой IO.
\end{frame}

\begin{frame}
\frametitle{Классические фичи}
\begin{itemize}
\item Concurrency
\item Fault Tolerance
\item Distribution
\item Hot Code Upgrade
\end{itemize}
\end{frame}

\begin{frame}
\frametitle{Не классические, но очень полезные фичи}
\begin{itemize}
\item Symmetric Multiprocessing
\item Actor Model
\item Soft Real Time
\item Garbage Collection
\item Tracing
\end{itemize}
\end{frame}

\begin{frame}
\frametitle{Concurrency}
\centering
Процессы являются базовой сущностью языка.
\par \bigskip
\end{frame}

\begin{frame}
\frametitle{Concurrency}
\centering
Процессы легковесны,
\par
их можно создавать десятки и сотни тысяч.
\par \bigskip
\textcolor{blue}{1024 - 134,217,727 \begin{math}(2^{10} - 2^{27})\end{math}}
\par \bigskip
дефолтное значение \textcolor{blue}{262,144 \begin{math}(2^{18})\end{math}}
\end{frame}

\begin{frame}
\frametitle{Concurrency}
\centering
Запуск нового процесса - \textcolor{blue}{3-5} микросекунд.
\par \bigskip
На старте поток занимает \textcolor{blue}{2696} байт,
\par
включая стек, кучу и память под свои метаданные.
\end{frame}

\begin{frame}
\frametitle{Concurrency}
\centering
Нет разделяемой области памяти,
\par \bigskip
каждый процесс имеет свою изолированную память.
\end{frame}

\begin{frame}
\frametitle{Concurrency}
\centering
Ошибки в процессах также изолированы,
\par \bigskip
падение одного процесса не влияет на работу остальных.
\end{frame}

\begin{frame}
\frametitle{Fault Tolerance}
\centering
Три уровня обработки ошибок.
\end{frame}

\begin{frame}
\frametitle{Fault Tolerance}
\centering
Перехват исключений
\par \bigskip
Supervisor
\par \bigskip
Кластер
\end{frame}

\begin{frame}
\frametitle{Distribution}
\centering
Горизонтальное масштабирование.
\par \bigskip
Устойчивость в том числе и к аппаратным авариям.
\end{frame}

\begin{frame}
\frametitle{Location Transparency}
\centering
Процессы общаются отправкой сообщений друг другу.
\par \bigskip
При этом не важно, находятся ли они на одном узле,
\par \bigskip
или на разных.
\end{frame}

\begin{frame}
\frametitle{Hot Code Upgrade}
\centering
VM позволяет загрузить в рантайм новую версию кода модуля,
\par \bigskip
и переключить выполнение процесса
\par \bigskip
со старой версии кода на новую,
\par \bigskip
сохранив состояние памяти процесса.
\end{frame}

\begin{frame}
\frametitle{Symmetric Multiprocessing}
\centering
Запускается несколько планировщиков,
\par
соответственно количеству процессорных ядер.
\par \bigskip
Каждый планировщик использует один процесс ОС,
\par
и поверх него запускает эрланговские процессы.
\par \bigskip
Планировщики умеют балансировать нагрузку,
\par
перераспределяя потоки между собой.
\end{frame}

\begin{frame}
\frametitle{Symmetric Multiprocessing}
\centering
VM линейно масштабируется на большое количество ядер.
\par \bigskip
Проверено на практике на машинах с \textcolor{blue}{1024} ядрами.
\end{frame}

\begin{frame}
\frametitle{Actor Model}
\centering
Система состоит из акторов, которые действуют паралельно
\par
и независимо друг от друга.
\par \bigskip
Акторы общаются друг с другом
\par
с помощью отправки сообщений (message passing).
\par \bigskip
Данные копируются, поток не может изменить
\par
данные другого потока.
\par \bigskip
Отправка сообщений является асинхронной.
\end{frame}

\begin{frame}
\frametitle{Soft Real Time}
\centering
VM позволяет строить системы реального времени.
\par \bigskip
То есть, системы, где требуется предсказуемое время ответа.
\end{frame}

\begin{frame}
\frametitle{Soft Real Time}
\begin{itemize}
\item вытесняющая многозадачность (preemptive scheduling)
\item настраиваемый IO
\item особенности сборки мусора (garbage collection)
\end{itemize}
\end{frame}

\begin{frame}
\frametitle{Garbage Collection}
\centering
Сборка мусора в ФП проще, благодаря иммутабельным данным.
\par \bigskip
GC использует обычный алгоритм с двумя поколениями данных.
\par \bigskip
Но есть важная особенность...
\end{frame}

\begin{frame}
\frametitle{Garbage Collection}
\centering
Отдельная сборка мусора для каждого процесса.
\par \bigskip
Блокирует только один процесс.
\par \bigskip
Срабатывает быстро.
\par \bigskip
Нет эффекта \textbf{stop world}, характерного для JVM.
\end{frame}

\begin{frame}
\frametitle{Tracing}
\begin{itemize}
\item жизненный цикл процессов
\item отправка и получение сообщений
\item вызовы функций, аргументы, возвращаемые значения
\item состояние процессов
\item работа планировщика
\item потребление памяти
\item работа сборщиков мусора
\end{itemize}
\end{frame}

\begin{frame}
\frametitle{Tracing}
\centering
Можно узнать почти все о работе ноды.
\par \bigskip
Сложность в том, чтобы
\par
собрать именно ту информацию, которая нужна.
\end{frame}

\begin{frame}
\frametitle{Языки для Erlang VM}
\centering
Erlang, Elixir
\par \bigskip
Joxa, LFE
\par \bigskip
Alpaca, Gleam
\par \bigskip
\textcolor{blue}{\href{https://github.com/llaisdy/beam_languages}{и другие}}
\end{frame}


\section{Эрланг}

{
\setbeamercolor{background canvas}{bg=orange}
\begin{frame}
\frametitle{Какие основные фишки у Эрланг.}
\centering
Сейчас откровется вся правда.
\end{frame}
}

\begin{frame}
\frametitle{Эрланг}
\centering
Простой язык, который можно быстро освоить.
\par \bigskip
Консервативный, в него не часто добавляются новые фичи.
\par \bigskip
При этом быстро развивается виртуальная машина.
\par \bigskip
(Язык и виртуальную машину развивает одна команда).
\end{frame}

\begin{frame}
\frametitle{Эрланг}
\centering
Язык водопроводчиков.
\par \bigskip
Школьные задачи про трубы и бассейны.
\par \bigskip
В центре внимания: потоки данных и хранилища данных.
\par \bigskip
RabbitMQ и Riak.
\end{frame}

\begin{frame}
\frametitle{Эрланг}
\centering
Язык не про то, чтобы создать сложные абстракции,
\par
сложные модели данных,
\par
и сложные взаимодействия между ними.
\par \bigskip
(Для этого в Эрланг мало выразительных средств).
\par \bigskip
Язык про то, чтобы эффективно использовать ресурсы.
\end{frame}

\begin{frame}
\frametitle{Эрланг}
\centering
Основные проблемы, которые решают эрлангисты:
\par \bigskip
нехватка ресурсов,
\par \bigskip
не использование имеющихся ресурсов,
\par \bigskip
обе проблемы одновременно.
\end{frame}

\begin{frame}
\frametitle{Эрланг}
\centering
Типичная задача:
\par \bigskip
Пропускная способности системы ниже, чем нужно,
\par
но системе есть свободные ресурсы:
\par
CPU, память, IO.
\par \bigskip
Найти узкое место.
\par \bigskip
Устранить его.
\end{frame}

\begin{frame}
\frametitle{Эрланг}
\centering
Сильная сторона:
\par \bigskip
способность держать одновременно
\par
много \textbf{долгоживущих} соединений/сессий.
\end{frame}

\begin{frame}
\frametitle{Эрланг}
\centering
Типичный пример: чат-сервер.
\par \bigskip
Сотни тысяч одновременных подключений.
\par \bigskip
Длительность сессии -- от десятков минут до нескольких часов.
\end{frame}

\begin{frame}
\frametitle{Эрланг и веб}
\centering
Веб-сервер: Cowboy
\par \bigskip
Веб-фреймворк: нет
\par \bigskip
ORM: нет (мы пишем SQL руками)
\end{frame}

%% В эрланг проектах обычно схема данных не сложная, 5-10 таблиц, 10-20 вариантов запросов к ним, и все.
%% SQL пишется вручную, и это не доставляет проблем.

%% Сложность растет не в сторону 100 сущностей, 50 связей между ними, а в сторону шардирования таблиц.

%% Схема WGNC: email_notification, 32 шарда, дневные партиции. Запросы в конкретную партицию. ORM в таких делах не помогает.
%% В эликсировских проектах Ecto не расчитана на шардирование, и тут есть сложности, которые нужно решать отдельно.

\begin{frame}
\frametitle{Эрланг и веб}
\centering
Бэкенд сервис, API для других сервисов.
\par \bigskip
Не работает с UI.
\par \bigskip
Не работает непосредственно с пользователем.
\end{frame}


\section{Какие основные фишки у Эликсир.}

{
\setbeamercolor{background canvas}{bg=orange}
\begin{frame}
\frametitle{Какие основные фишки у Эликсир.}
\centering
Срываем покровы.
\end{frame}
}

\begin{frame}
\frametitle{Эликсир}
\centering
Может делать все, что может Эрланг.
\par \bigskip
И может больше,
\par
закрывает слабые стороны Эрланг.
\end{frame}

\begin{frame}
\frametitle{Эликсир}
\centering
Мощный, выразительный язык.
\par \bigskip
Можно создавать сложные абстракции,
\par
сложные модели данных,
\par
сложные взаимодействия между ними.
\end{frame}

\begin{frame}
\frametitle{Эликсир}
\centering
Расширяемый,
\par
имеет богатые средства метапрограммирования.
\par \bigskip
Веб-фреймворк, ORM -- в наличии.
\par \bigskip
Библиотеки (hex), тулинг (iex, mix).
\end{frame}

\begin{frame}
\frametitle{Эликсир}
\centering
В некотором роде -- противоположность Эрланг.
\par \bigskip
И многие эрлангисты его не любят.
\par \bigskip
Я тоже принял Эликсир не сразу, долго сопротивлялся :)
\end{frame}


\section{Разбор мнений об Эликсир, бытующих в Руби-сообществе.}

{
\setbeamercolor{background canvas}{bg=orange}
\begin{frame}
\frametitle{Разбор мнений об Эликсир, бытующих в Руби-сообществе.}
\centering
Правда или миф?
\end{frame}
}

\begin{frame}
\frametitle{Разбор мнений об Эликсир}
\centering
\textcolor{blue}{Рубисту просто начать писать на Эликсире,}
\par
\textcolor{blue}{так как синтаксис очень похож.}
\end{frame}

\begin{frame}
\frametitle{Разбор мнений об Эликсир}
\centering
\textcolor{blue}{Рубисту просто начать писать на Эликсире,}
\par
\textcolor{blue}{так как синтаксис очень похож.}
\par \bigskip
\textbf{Неправда}.
\par \bigskip
Нужно освоить многие новые сложные концепции:
\par
ФП, OTP, акторная модель и т.д.
\par \bigskip
Синтаксис имеет мало значения.
\par
Да он и не похож.
\end{frame}

%% Ну синтаксис может быть местами и похож. Была такая идея на ранних этапах развития Эликсир,
%% сделать его похожим по синтаксису на Руби. Потом от этого отказались.

%% Дело в том, что сам по себе синтаксис не важен, важна семантика. А семантика у Эликсир совсем другая, чем у Руби.

%% В основе эликсир две ключевые вещи: функциональная парадигма программирования, и акторная модель многопоточности.
%% Их нужно будет понять и принять, и синтаксис тут ничем не поможет.

%% Что касается легкости изучения, то из моего опыта (я обучаю эрлангу уже давно, эликсиру пока еще нет, готовлюсь) бывают разные ситуации.
%% Если человек уже знаком с ФП, то проблем не возникает.
%% Если не знаком, то как повезет. Кто-то с ходу все схватывает, и тут же пишет правильный код. Кому-то никак не заходит, и он сдается.

%% Нормально работает вариант, когда в команде есть 1-2 эксперта, а остальные переучились на ЭЭ с других языков.
%% Вариант, когда экспертов нет, а есть только переученые люди с опытом в других языках -- слишком рискованый, я не рекомендую.

\begin{frame}
\frametitle{Разбор мнений об Эликсир}
\centering
\textcolor{blue}{Эликсир хорошо масштабируется.}
\par \bigskip
Если пишешь приложение на Эликсире, то DevOps не нужен.
\par
Будет и так хорошо работать и масштабироваться.
\par
Нужно просто инстансов больше закидывать.
\end{frame}

\begin{frame}
\frametitle{Разбор мнений об Эликсир}
\centering
\textcolor{blue}{Эликсир хорошо масштабируется.}
\par \bigskip
\textbf{Правда}.
\par \bigskip
Но не сам по себе, волшебным образом.
\par
А для этого нужно подумать и поработать.
\end{frame}

%% Если вы ждете, что масштабирование работает само по себе из коробки, и думать ни о чем не надо, но это неправда.

%% Масштабирование -- архитектурная проблема, и она ортогональна языку программирования.
%% Язык можно использовать любой. А думать нужно, много, о разном.
%% (Шардирование, батчинг, репликация, кэши, консистентность данных и т.д.)

%% Чем отличается ЭЭ, это тем, что накоплено 40 лет опыта масштабирования таких систем, есть инструменты, описаны подходы.

\begin{frame}
\frametitle{Разбор мнений об Эликсир}
\centering
\textcolor{blue}{Эликсир хорош для работы с WebSocket.}
\end{frame}

\begin{frame}
\frametitle{Разбор мнений об Эликсир}
\centering
\textcolor{blue}{Эликсир хорош для работы с WebSocket.}
\par \bigskip
\textbf{Правда}.
\par \bigskip
Способность держать одновременно
\par
много \textbf{долгоживущих} соединений/сессий.
\end{frame}

%% И это такая ниша, где у ЭЭ мало конкурентов.

\begin{frame}
\frametitle{Разбор мнений об Эликсир}
\centering
\textcolor{blue}{Эликсир работает быстрее чем Руби.}
\end{frame}

\begin{frame}
\frametitle{Разбор мнений об Эликсир}
\centering
\textcolor{blue}{Эликсир работает быстрее чем Руби.}
\par \bigskip
\textbf{Правда}.
\par \bigskip
Но тут есть много о чем поговорить:
\par \bigskip
Сферическая скорость в вакууме, а зачем? Экономить железо?
\par
Есть конкретные проблемы, например, с латенси запросов?
\par
Важна скорость или производительность?
\par
А производительность, это что вообще такое?
\end{frame}

%% Ох, какая большая тема для обсуждения!

%% Это правда. Сферическая скорость в вакууме для ЭЭ будет больше, чем для питона и руби, но медленнее жава и го.
%% Пожалуй, и не только сферическая, и а в большинстве прикладных случаев тоже.
%% Но дело в том, это не так важно, как принято считать.
%% И идея и архитектура Эрланговской виртуальной машины, она не про скорость. Она про максимально эффективное использование ресурсов.

%% У нас как-то принято много обсуждать производительсность.
%% Часто можно видеть обсуждения, что вот такая конструкция языка работает быстрее, чем эдакая. И этому посвящают немало времени.
%% На самом деле вот эта вот производительность на уровне алгоритмов и структур данных переоценена.

%% Да, для бизнеса важна производительность. Но обычно проблемы производительности, важные для бизнеса, решаются на уровне архитектуры.
%% Да, есть отдельные области, где нужно выжать максимум из тактов процессора, байтов памяти и пропускной способности сети:
%% - микроконтроллеры и другое маломощное встраиваемое железо
%% - биржевая торговля
%% - база данных

%% Но гораздо чаще нас интересует масштабирумость системы, а не экономия железа.
%% Железо стоит дешевле, чем время программистов. А вот если сервис не масштабируется, тогда это проблема, которую нужно решать, и ЭЭ решают ее неплохо.

%% Но если вам действительно нужно экономить железо, то для ЭЭ есть некоторые подходы.

%% Правда в некоторых комментариях я видел, что речь идет не о скорости абстрактной в вакууме,
%% А конкретно о латерси в ответах на HTTP-запросы.
%% Обычно проблемы латенси HTTP-сервиса сводятся к проблемам латенси запросов в БД.
%% И это либо уровень приложения (ORM генерирует неэффективные запросы, либо выполняет больше запросов чем нужно),
%% либо уровень самой БД.
%% Ecto -- явное разделение генерации запроса и выполнения запроса. Можно выполнять сырой SQL.

%% А есть еще конфликт latency vs throuput. И тут тоже есть о чем поговорить.


\begin{frame}
\frametitle{Разбор мнений об Эликсир}
\centering
\textcolor{blue}{Эликсирщикам больше платят.}
\end{frame}

\begin{frame}
\frametitle{Разбор мнений об Эликсир}
\centering
\textcolor{blue}{Эликсирщикам больше платят.}
\par \bigskip
\textbf{Не знаю}
\par \bigskip
Эксперт с редкой специализацией часто стоит дороже.
\par \bigskip
При этом ему труднее найти работу.
\end{frame}

%% Тут трудно сказать. Вероятно, зависит от локации, и многих других факторов.
%% https://salaries.dev.by/ -- данных для эрланг/эликсир нет

%% Но из общих соображений можно сказать следущее:
%% Эксперт с редкой специализацией часто стоит дороже, чем эксперт с широко распространенной специализацией.

%% Но тут есть два НО:
%% - Эксперт с профильным опытом.
%% - Редкая специализация нужна не многим компаниям, поэтому могут быть трудности с поиском работы.

%% Трудности в обе стороны: компании трудно найти инженера, инженеру трудно найти компанию.
%% Такой ситуации, когда вообще нет вариантов работы, может и не будет. Но вот выбор вариантов будет гораздо уже.
%% Если вам надоели рекрутеры в линкедин, идите в эликсир, и тогда они перестанут надоедать :)

%% Хорошо быть экспертом и широкого профиля, и узкого одновременно :) Хватило бы жизни, чтобы таким стать :)


\begin{frame}
\frametitle{Разбор мнений об Эликсир}
\centering
\textcolor{blue}{Учить эликсир можно не ради того,}
\par
\textcolor{blue}{чтобы писать на нем в продакшн.}
\end{frame}

\begin{frame}
\frametitle{Разбор мнений об Эликсир}
\centering
\textcolor{blue}{Учить эликсир можно не ради того,}
\par
\textcolor{blue}{чтобы писать на нем в продакшн.}
\par \bigskip
\textbf{Да}
\par \bigskip
Изучать новые концепции и идеи всегда полезно
\par
для расширения кругозора.
\end{frame}


%% Имо, основная ценность изучения эликсира рубисту - посмотреть как задизайнен OTP и как устроен Phoenix.
%% Если написать даже простейший бложик на фениксе, то есть шанс, что рейлс головного мозга отомрет быстрее
%% и будет больше мотивации потрогать какой-нибудь Hanami.

%% Соглашусь. Изучать разные ЯП и разные подходы к программированию полезно для расширения кругозора.

%% На мой взгляд хороший инженер должен иметь 2-3 ЯП в своем арсенале, и уметь выбирать подходящий для конкретного проекта.
%% В варгейминге именно это и практикуется.


\begin{frame}
\frametitle{Разбор мнений об Эликсир}
\centering
\textcolor{blue}{Code aesthetics and language UX.}
\par \bigskip
Elixir is the first programming language after Ruby
\par \bigskip
that considers code aesthetics and language UX.
\end{frame}


\begin{frame}
\frametitle{Разбор мнений об Эликсир}
\centering
\textcolor{blue}{Code aesthetics and language UX.}
\par \bigskip
Не могу согласиться :)
\par \bigskip
Автор этого мнения, вероятно, не знаком с Haskell или OCaml.
\end{frame}

%% Те, кто знаком с Haskell или OCaml, увидят, что синтаксис в Elixir более громозкий и менее последовательный, чем в этих языках.
%% Дизайн языка сформировался во многом случайно, а не по плану.
%% Впрочем, это не так важно. Во-первых, эстетика -- дело вкуса. Во-вторых, будешь долго работать с языком, привыкнешь, и перестанешь замечать.
%% Детали можно обсудить в кулуарах.

%% Примеры косяков в дизайне языка:
%% - точка при вызове анонимной функции
%% - неудавшаяся попытка реализовать без-скобочный синтаксис вызова функции
%% - конструкция with, где первое выражение обязательно должно быть на той же строке, что и with
%% - три вида исключений и много вариантов их обработки. (Слава богу, ни один из них почти никогда не нужен).
%% - pipe-оператор подставляет первый аргумент, а не последний, как во всех других ML-языках. (Это из-за отсутствия каррирования).

{
\setbeamercolor{background canvas}{bg=orange}
\begin{frame}
\centering
А теперь, каверзные \textbf{вопросы}, ставящие докладчика в тупик.
\par \bigskip
Ну и не каверзные тоже можно.
\end{frame}
}

\end{document}
